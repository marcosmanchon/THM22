\documentclass[crop=false]{standalone}
\usepackage{graphicx}
\graphicspath{{images/}}
\usepackage[utf8]{inputenc}
\usepackage{blindtext}

\begin{document}


\begin{figure}
\centering
\includegraphics[width=3cm]{images/Escudo_UNCuyo.png}
\label{fig:lionfigure}
\caption{Escudo Oficial propuesto por el Rectorado de la UNCuyo}
\end{figure}

Según Ord 38/93 del Consejo Superior todos los trabajos que se realicen en el ámbito de la secretaría académica deben llevar los logos e isotipos que se establecen como MARCA de la UNCuyo. 
Ver detalles en: https://cicunc.uncuyo.edu.ar/imagen-institucional

\section{Introducción}
En la era actual de transición hacia fuentes de energía más limpias y sostenibles, el litio se ha convertido en un recurso estratégico de gran relevancia, desempeñando un papel fundamental en la revolución de la movilidad eléctrica y el almacenamiento de energía. Argentina, con sus vastas reservas de litio, se encuentra en una posición única para capitalizar esta creciente demanda global y contribuir significativamente a la transformación del panorama energético mundial.

El presente proyecto tiene como objetivo principal la industrialización de litio en Argentina, mediante la creación de una cadena de valor integral que abarque desde la extracción sostenible de litio hasta la producción de productos de alto valor agregado, como baterías de ion-litio y otros componentes esenciales para la electromovilidad y el almacenamiento de energía renovable. Esta iniciativa no solo busca aprovechar las ventajas competitivas del país en términos de recursos minerales, sino también promover el desarrollo económico y tecnológico a nivel nacional, generando empleo de calidad y potenciando la innovación en el sector industrial.

La industrialización del litio no solo implica una oportunidad económica, sino también un compromiso con la sustentabilidad ambiental y social. Es imperativo que el desarrollo de la cadena de valor del litio se realice de manera responsable y respetuosa con el entorno, mediante la implementación de prácticas de extracción y producción que minimicen los impactos ambientales y fomenten el bienestar de las comunidades locales.

A lo largo de esta formulación de proyecto, se presentarán los objetivos específicos, la descripción detallada de las etapas involucradas en la industrialización del litio, el análisis de viabilidad económica y financiera, así como las estrategias de mitigación de riesgos y los potenciales impactos positivos en la economía y sociedad argentinas. Este proyecto aspira a consolidarse como un pilar fundamental en la diversificación de la matriz productiva del país y en la construcción de un futuro energético más limpio, sostenible y promisorio.

En resumen, la industrialización de litio en Argentina representa una oportunidad estratégica para avanzar hacia un modelo económico basado en la innovación, la sostenibilidad y la creación de valor agregado, mientras contribuye activamente a la mitigación del cambio climático y al impulso de la industria energética global.

\textbf{Este texto ha sido generado por Inteligencia Artificial}







%\blindtext

\end{document}

